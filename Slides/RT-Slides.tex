\documentclass[slidestop,compress,mathserif, aspectratio = 169]{beamer}
%\usecolortheme{seagull} % Beamer color theme
\usetheme{metropolis}
%\usepackage{helvet}
\useinnertheme{rectangles}
\usepackage{bibentry}
\usepackage{graphicx}
\usepackage{epigraph}
\usepackage{makeidx}
\makeindex
\usepackage[]{amsmath}
\usepackage[]{amssymb}
\usepackage{color}
\usepackage{pict2e}
\usepackage{algorithm2e}
\usepackage[ngerman]{babel}
\usepackage{tikz}
\usepackage[europeanresistors,europeaninductors]{circuitikz}
\usepackage{multirow}
\usepackage[utf8]{inputenc}
\usepackage{multimedia}
\usepackage{xcolor}
\usepackage{bm}
\usetikzlibrary{arrows, decorations.markings}
\usepackage{pgfplots}
\usetikzlibrary{shapes.misc}
\tikzset{cross/.style={cross out, draw=black, minimum size=2*(#1-\pgflinewidth), inner sep=0pt, outer sep=0pt},
%default radius will be 1pt. 
cross/.default={1pt}}

\tikzstyle{vecArrow} = [thick, decoration={markings,mark=at position
   1 with {\arrow[semithick]{open triangle 60}}},
   double distance=1.4pt, shorten >= 5.5pt,
   preaction = {decorate},
   postaction = {draw,line width=1.4pt, white,shorten >= 4.5pt}]
\tikzstyle{innerWhite} = [semithick, white,line width=1.4pt, shorten >= 4.5pt]



\definecolor{blue}{rgb}{0,0,1}
\definecolor{mint}{cmyk}{75,0,40,0}
\definecolor{mint}{rgb}{32,178,170}
\begin{document}
% Formalia
\newcommand{\revision}{Rev. D}
\newcommand{\docnum}{SFV-14031}
\newcommand{\done}{${\color{teal}\checkmark}$}
\newcommand{\Var}{\operatorname{Var}}
\newcommand{\D}{\operatorname{d}}
\newcommand{\mum}{\operatorname{\mu m}}
\newcommand{\E}{\operatorname{E}}
\newcommand{\var}{\operatorname{var}}
\newcommand{\Id}{\operatorname{Id}}
\newcommand{\rg}{\operatorname{rg}}

\graphicspath{{../../figures/}}



\newcommand{
\offslide}[2]{
\begin{frame}
\frametitle{\includegraphics[scale=0.05] {Offwhite} \hspace{.5mm} #1}
%\framesubtitle{#2}
\begin{tikzpicture}[x=1cm, y=1cm, semitransparent]
\draw[step=5mm, line width=0.2mm, black!40!white] (0,0) grid (\textwidth,\textheight-1cm);
\node[anchor = south east] at (\textwidth,0) {#2};

\end{tikzpicture}
\end{frame}
}

\newcommand{\source}[1]{\rotatebox{90}{\tiny \color{gray} #1}}


%\newtheorem{lemma}{Lemma}
\title{Regelungstechnik}
%\date{}
\subtitle{Kurze Einf\"uhrung in die Zustandsraumdarstellung}
\author[R. Pfaff]{Prof. Dr. Raphael Pfaff}
\institute[Dokument \docnum, \revision]{Fachhochschule Aachen}
\logo{\put(-18, 120){\includegraphics[width=1.4cm]{logoR}}}

\begin{frame} % Cover slide
\titlepage
\end{frame}
\section{Einf\"uhrung Zustandsraum}
%\frame{\sectionpage}
\frame{\frametitle{Einf\"uhrung Zustandsraum \textit{(state space)}}
\framesubtitle{Der Zustandsraum ist eine zunehmend an Bedeutung gewinnende Darstellung eines dynamischen Systems.}
\begin{columns}[t] 
     \begin{column}[T]{6cm} 
     	\begin{itemize}
     		\item Zustandsvariablen $x_{i}(t)$:
		\begin{itemize}
		\item Energiegehalt der Speichersysteme
		\item Bilden Zustandsvektor $x(t)$
		\end{itemize}
		\item $x(t_{0})$ enth\"alt alle Informationen \"uber System zum Zeitpunkt $t_{0}$
		\item \"Uberf\"uhrung der DGL $n$-ter Ordnung in $n$ DGL 1. Ordnung
		\item Hier betrachtet: \textit{single-input-single-output} (SISO) System
     	\end{itemize}
     \end{column}
     	\begin{column}[T]{6cm} 
		\textbf{Zustandsraumdarstellung:}
         	\begin{block}{Zeitkontinuierlich}
		\begin{eqnarray}
			\dot{x}(t) &= A x(t) + B u(t) \label{Eq:SSCT1}\\
			y(t) &= Cx(t) + D u(t) \label{Eq:SSCT2}
		\end{eqnarray}
		\end{block}
		\begin{block}{Zeitdiskret}
         	\begin{eqnarray}
			x_{k+1} &= A x_{k} + B u_{k} \label{Eq:SSDT1}\\
			y_{k} &= Cx_{k} + D u_{k} \label{Eq:SSDT2}
		\end{eqnarray}
		\end{block}
     \end{column}
 \end{columns}
}

\frame{\frametitle{Bedeutung der Variablen und Parameter}
\framesubtitle{Gr\"o{\ss}en der Variablen und Parameter f\"ur SISO System der Ordnung $n$, $\cdot(t)$ beschreibt Variablen, d.h. $\cdot_{k}$ bzw. $\cdot(t)$.}
\begin{columns}[t] 
     \begin{column}[T]{6cm} 
     	\begin{itemize}
     		\item $A \in \mathbb{R}^{n \times n}$: Systemmatrix
		\begin{itemize}
		\item Systeminformationen, z.B. Polstellen, Beobachtbarkeit, ...
		\end{itemize}
		\item $B \in \mathbb{R}^{n \times 1}$: Eingangsvektor
		\begin{itemize}
		\item Wirkung von $u$ auf System
		\end{itemize}
		\item $C \in \mathbb{R}^{1 \times n}$: Ausgangsvektor
		\begin{itemize}
		\item Wirkung von $x$ auf Ausgang
		\end{itemize}
		\item $D \in \mathbb{R}$: Feedforward term
		\begin{itemize}
		\item Direkte Wirkung von $u$ auf Ausgang
		\end{itemize}
     	\end{itemize}
     \end{column}
     	\begin{column}[T]{6cm} 
         	\begin{eqnarray*}
			\dot{x}(t) &= A x(t) + B u(t) \\
			y(t) &= Cx(t) + D u(t) \\ \vspace{.5cm}
%			x_{k+1} &= A x_{k} + B u_{k} \\
%			y_{k} &= Cx_{k} + D u_{k}
		\end{eqnarray*}
		\begin{itemize}
		\item $x(t) \in \mathbb{R}^{n \times 1}$: Zustandsvektor
		\begin{itemize}
		\item Zustand der Energiespeicher
		\end{itemize}
		\item $u(t) \in \mathbb{R}$: Eingangswert in System
		\item $y(t) \in \mathbb{R}$: Ausgangswert des Systems
		\end{itemize}
     \end{column}
 \end{columns}
}

\frame{\frametitle{Blockdiagramm f\"ur SISO-System}
\framesubtitle{}
\begin{center}
\begin{tikzpicture}[thick, node distance = 1.75cm]
  \node[draw,rectangle] (i) {$\int$};
   \node[draw, circle, left of=i] (l){};
  \node[draw, rectangle, left of=l] (b){$\bm{B}$};
  \node[draw,rectangle,below of=i] (a) {$\bm{A}$};
  \node[draw,rectangle,above of=i] (d) {$\bm{D}$};
      \node[inner sep=0,minimum size=0,right of=i] (k) {}; 
        \node[draw,rectangle,right of=k] (c) {$\bm{C}$};
    \node[inner sep=0,minimum size=0,left of=b] (p) {}; 
    \node[inner sep=0,minimum size=0,left of=p] (o) {};
   \node[draw, circle,right of=c] (m) {}; 
   \node[inner sep=0,minimum size=0,right of=m] (q) {};
  
  % 1st pass: draw arrows
  \draw[vecArrow] (i) -- (c) node[near start, above] {$\bm{x}(t)$};
  \draw[vecArrow] (k) |- (a);
  \draw[vecArrow] (a) -| (l);% node[pos = 0.95, right] {$+$};
  \draw[vecArrow] (l) -- (i) node[midway, above] {$\dot{\bm{x}}(t)$};
  \draw[vecArrow] (b) to (l); %node[pos = 0.95, below] {$+$};
  \draw[thick, -stealth] (o) -- (b) node[pos = 0.25, above] {$u(t)$};
   \draw[thick, -stealth] (p) |- (d);
    \draw[thick, -stealth] (d) -| (m);
       \draw[thick, -stealth] (c) -- (m);
        \draw[thick, -stealth] (m) -- (q) node[pos = 0.5, above] {$y(t)$};
  
\end{tikzpicture}

\end{center}
}


\frame{\frametitle{Vorteile/Nachteile Zustandsraumdarstellung}
\framesubtitle{}
\begin{itemize}
\item[+] Reichhaltige Beschreibung: interne Werte des Systems werden modelliert
\item[+] Vollst\"andige Abbildung des Systemzustands durch die $x_{i}$
\item[+] Digitale Regler im Zustandsraum performanter
\item[+] F\"ur Energieeffizienz: Energiegehalt des System wird modelliert
\item[+] Zustand hat h\"aufig physikalische Bedeutung
\item[-] Zustandsraumdarstellung nicht eindeutig
\item[-] Teils numerisch/analytisch aufwendig
\end{itemize}
}
\section{Normalformen}

%\frame{\frametitle{Zustandsraumdarstellung RLC-System}
%\framesubtitle{}
%         \begin{center}
%		\begin{circuitikz}[scale = 0.8]
%		\draw 
%	(-2,0) to  [V=$u(t)$] (-2,4)
%		to [R=$R$,  i>^=$i(t)$] (2,4)
%		to [L=$L$] (6,4)
%		to [C=$C$, v_>=$u_{C}(t)$] (6,0)
%  		 to [short] (-2,0);
%		 \end{circuitikz}
%        	\end{center}
%	\vspace{-1cm}
%	\begin{columns}[t] 
%     \begin{column}[T]{.5cm} 
%     \end{column}
%     	\begin{column}[T]{10cm} 
%         	\begin{center}
%%	\only<1>{
%%%	\begin{block}{\"Ubung 1}
%%%	Stellen Sie eine Zustandsraumbeschreibung f\"ur den Zustandsvektor $\begin{pmatrix} 
%%%		u_{C}(t) \\ \frac{1}{C} i(t)
%%%		\end{pmatrix}$ auf.
%%%		\end{block}
%%	}
%%	\only<2>{
%	\begin{eqnarray}
%		\dot{x}(t) &=& 
%		\begin{pmatrix} 
%		0 & 1 \\ \frac{-1}{L C} & \frac{-R}{L}
%		\end{pmatrix}
%		\begin{pmatrix} 
%		x_{1}(t) \\ x_{2}(t)
%		\end{pmatrix} +
%		\begin{pmatrix} 
%		0 \\ \frac{1}{LC} 
%		\end{pmatrix} u(t) \\
%		y(t) &=& u_{C}(t) = \begin{pmatrix} 
%		1 & 0
%		\end{pmatrix}
%		 \begin{pmatrix} 
%		x_{1}(t) \\ x_{2} (t)
%		\end{pmatrix}
%	\end{eqnarray}
%%}
%\end{center}
%     \end{column}
% \end{columns}
%}



\frame{\frametitle{Regelungsnormalform \textit{(control canonical form)}}
	\begin{equation*}
		\begin{split}
		&\frac{\D^{n} y}{\D t^{n}} + a_{n-1} \frac{\D^{n-1} y}{\D t^{n-1}} + \cdots + a_{1} \frac{\D y}{\D t} + a_{0} y  \\
		&= b_{n} \frac{\D^{n} u}{\D t^{n}} + b_{n-1} \frac{\D^{n-1} u}{\D t^{n-1}} + \cdots + b_{1} \frac{\D u}{\D t} + b_{0} u 
		\end{split}
	\end{equation*}
\begin{columns}[t] 
     \begin{column}[T]{8cm} 
     	\begin{eqnarray*}
            A&=& \begin{pmatrix}%{ccccc}
             0 & 1  &  0 & \cdots & 0 \\
             0 & 0  &  1 & \cdots & 0 \\
            \vdots &  \vdots &  \vdots & \ddots & \vdots \\
            0 & 0 & 0 & \cdots & 1 \\
            -a_{0} & -a_{1} & -a_{2} &\cdots & -a_{n-1}   
            \end{pmatrix} \\
            C &=& \begin{pmatrix} b_{0} - b_{n}a_{0} & b_{1} - b_{n}a_{1} & \cdots & b_{n-1} - b_{n}a_{n-1} \end{pmatrix}
	\end{eqnarray*}
     \end{column}
     	\begin{column}[T]{3cm} 
         	\begin{eqnarray*}
                 B &=& \begin{pmatrix} 0 \\ 0 \\ \vdots \\ 0 \\ 1\end{pmatrix} \\
                 D &=& b_{n}%\footnote{\"Ublich: $b_{n} = 0$}
	\end{eqnarray*}
     \end{column}
 \end{columns}
}

%\offslide{Darstellung Regelungsnormalform im Blockdiagramm}

\frame{\frametitle{Beobachtungsnormalform \textit{(observer canonical form)}}
\begin{equation*}
		\begin{split}
		&\frac{\D^{n} y}{\D t^{n}} + a_{n-1} \frac{\D^{n-1} y}{\D t^{n-1}} + \cdots + a_{1} \frac{\D y}{\D t} + a_{0} y  \\
		&= b_{n} \frac{\D^{n} u}{\D t^{n}} + b_{n-1} \frac{\D^{n-1} u}{\D t^{n-1}} + \cdots + b_{1} \frac{\D u}{\D t} + b_{0} u 
		\end{split}
	\end{equation*}
\begin{columns}[t] 
     \begin{column}[T]{7cm} 
     	\begin{eqnarray*}
            A&=& \begin{pmatrix}%{ccccc}
             0 & \cdots & 0 &  0  & -a_{0} \\
             1 & \cdots & 0  &  0  & -a_{1} \\
            \vdots &  \ddots & \vdots &  \vdots  & \vdots \\
            0 & \cdots & 1 & 0  &  -a_{n-2} \\
            0 & \cdots&  0 & 1  & -a_{n-1}   
            \end{pmatrix} \\
            C &=& \begin{pmatrix} 0 & 0 & \cdots & 0 & 1\end{pmatrix} 
	\end{eqnarray*}
     \end{column}
     	\begin{column}[T]{4cm} 
         	\begin{eqnarray*}
                 B &=&  \begin{pmatrix} b_{0} - b_{n}a_{0} \\ b_{1} - b_{n}a_{1} \\ b_{2} - b_{n}a_{2}\\ \vdots \\ b_{n-1} - b_{n}a_{n-1} \end{pmatrix}\\
                 D &=& b_{n}%\footnote{\"Ublich: $b_{n} = 0$}
	\end{eqnarray*}
     \end{column}
 \end{columns}
}

%\offslide{Darstellung Beobachtungsnormalfom im Blockdiagramm}

%\frame{\frametitle{\"Ubung 2: Transformation \"Ubertragungsfunktion in Zustandsraumdarstellung}
%Es sei \[G(s) = \frac{Y(s)}{U(s)} = \frac{s+3}{(s+1)(s+10)}\] die \"Ubertragungsfunktion eines Systems.
%\begin{block}{\"Ubung 2}
%Stellen Sie $G(s)$ dar als:
%\begin{enumerate}
%		\item Differentialgleichung
%		\item Zustandsraumdarstellung f\"ur einen Zustandsvektor $\begin{pmatrix} \dot{x} \\ x\end{pmatrix}$ in
%		\begin{enumerate}
%		\item Regelungsnormalform
%		\item Beobachtungsnormalform
%		\end{enumerate}
%		\end{enumerate}
%\end{block}
%}


\frame{\frametitle{Jordan Normalform \textit{(modal canonical form)}}
	\begin{itemize}
		\item F\"ur System mit Polstellen $\lambda_{i}$
		\item Transferfunktion $G(s) = \sum_{i=1}^{n} \frac{c_{i}}{s-\lambda_{i}} $ 
	\end{itemize}
\begin{columns}[t] 
     \begin{column}[T]{7cm} 
     	\begin{eqnarray*}
            A&=& \begin{pmatrix}%{ccccc}
              \lambda_{1}  &  0 & \cdots & 0 \\
              0  &  \lambda_{2} & \cdots & 0 \\
            \vdots &  \vdots & \ddots & \vdots \\
            0 & 0 &\cdots & \lambda_{n}   
            \end{pmatrix} \\
            C &=& \begin{pmatrix} c_{1} & c_{2} & \cdots  & c_{n}\end{pmatrix} 
	\end{eqnarray*}
     \end{column}
     	\begin{column}[T]{4cm} 
         	\begin{eqnarray*}
                 B &=&  \begin{pmatrix} 1 \\ 1\\ \vdots \\1 \end{pmatrix}\\
               	\end{eqnarray*}
     \end{column}
 \end{columns}
}
 \section{Richtungsfeld}
 
\frame{\frametitle{Das Richtungsfeld einer DGL}
\framesubtitle{}
\begin{columns}[t] 
     \begin{column}[T]{7cm} 
     	\begin{itemize}
     		\item DGL zweiten Grades bzw. System zweiter Ordnung
		\item Zu jedem Punkt $(x_{1}, x_{2})^T$ l\"asst sich die Steigung $(\dot{x}_{1}, \dot{x}_{2})^T$ bestimmen:
		\begin{equation*}
		\left(
		\begin{array}{c}
		\dot{x}_{1}  \\
		 \dot{x}_{2} 
		\end{array}
		\right) = 
		%
		\left(
		\begin{array}{cc}
		a_{12} & a_{12}  \\
		 a_{21} & a_{22}
		\end{array}
		\right) 
		%
		\left( \begin{array}{c} {x}_{1}  \\	{x}_{2} \end{array} \right)
		\end{equation*}
		\item Mit dem Richtungsfeld l\"asst sich eine L\"osung graphisch bestimmen
     	\end{itemize}
     \end{column}
     	\begin{column}[T]{7cm} 
         	\begin{center}
            		\only<1>{\includegraphics[width=\textwidth]{DirectionField}\source{}}
		\only<2>{\includegraphics[width=\textwidth]{DirectionFieldSolution}\source{}}
        		\end{center}
     \end{column}
 \end{columns}
}
 
\section{Eigenwerte und Polstellen}
\frame{\frametitle{Systemmatrix, Eigenwerte}
\framesubtitle{Es enth\"alt $A$ alle Informationen \"uber das Eigenverhalten des Systems.}
     	\begin{itemize}
     		\item Eigenwerte:
		\begin{itemize}
		\item K\"onnen Polstellen des Systems sein
		\item Ggf. K\"urzung gegen Nullstellen des Systems
		\item Bestimmung mittels charakterischem Polynom $\chi_{A}(\lambda)$
		\item Polstellen reell oder paarweise komplex
		\end{itemize}
     	\end{itemize}
	
         	\begin{equation}
			\begin{split}
			\chi_{A}(\lambda) &= \det\left(\lambda \Id - A\right) \\
			&= \det\left(\begin{pmatrix} \lambda & 0 & \cdots &0 \\
			0 & \lambda  & \cdots &0 \\ \vdots & \vdots &\ddots &\vdots \\
			0 & 0 & \cdots & \lambda \end{pmatrix} - 
			\begin{pmatrix} a_{1,1} & a_{1,2} & \cdots & a_{1,n} \\
			a_{2,1} & a_{2,2}  & \cdots & a_{2,n} \\ \vdots & \vdots &\ddots &\vdots \\
			a_{n,1} & a_{n,2} & \cdots & a_{n,n} \end{pmatrix}\right) = 0
			\end{split}
		\end{equation}
 }


 \frame{\frametitle{Matrixexponential}
\framesubtitle{Mittels des Matrixexponentials lassen sich lineare Differentialgleichungssystem (also Zustandsgleichungen) l\"osen.}
\begin{itemize}
\item Definiere f\"ur $A \in \mathbb{R}^{n\times n}$ die unendliche Reihe
\begin{equation}
\exp\left(A t\right) = \Id + At + A^{2} \frac{t^2}{2!} + A^{3} \frac{t^3}{3!} + \ldots
\end{equation}
\item Dann erf\"ullt
\begin{equation}
x(t) = \exp\left(At\right) x_{0} + \exp\left(At\right) \int_{0}^t \exp\left(-A t\right) B u(\tau) \D \tau
\end{equation}
die Differentialgleichung
\begin{equation}
\frac{\D}{\D t} x(t) = A x(t) + B u(t)
\end{equation}
\end{itemize}
}


\frame{\frametitle{Berechnung Matrixexponential}
\framesubtitle{Auch falls Simulationstechniken eingesetzt werden bringt die Kenntnis der qualitativen L\"osung Einblicke in das Systemverhalten.}
\begin{itemize}
\item F\"ur diagonalisierbare Matrizen $A = U D U^{-1}$, d.h. $\chi_{A}$ hat $n$ komplexe Nullstellen $\lambda_{i} \in \mathbb{C}$:
\begin{equation}
	\begin{split}
	\exp\left(A\right) = &U \exp\left(D\right) U^{-1} = \\
	&U 
	\begin{pmatrix} \exp\left(\lambda_{1} t \right) & 0 & \cdots &0 \\
	0 & \exp\left(\lambda_{2} t \right) &  \cdots& 0 \\ \vdots & \vdots & \ddots & \vdots \\
	0 & 0 & \cdots & \exp\left(\lambda_{n} t \right)   
	\end{pmatrix} U^{-1}
	\end{split}
\end{equation}
	
\end{itemize}
}

\frame{\frametitle{Interpretation von Polstellen (zeitkontinuierliche Systeme)}
\framesubtitle{}
\begin{center}
\begin{tikzpicture}[scale = .8]
 \draw[ultra thick,-stealth] (-7,0) -- (3,0) node[pos = 0.95, above] {$\Re$};
 \draw[ultra thick,-stealth] (0,-4) -- (0,5) node[pos = 0.95, right] {$\Im$};
 \begin{scope}[xshift=-6cm, yshift = 0cm] 
 \begin{axis}[thick, width=4cm, height = 4cm, yticklabels={}, xticklabels={}, axis x line=middle,
    axis y line=middle,domain = 0:5,
    axis background/.style={fill=white, opacity = 0.5}, anchor = center, ticks = none, axis line style={draw=gray, opacity = 0.7}, enlargelimits=true]
         \addplot[no markers, smooth]  
                  {exp(-5*x)};
        \node at (rel axis cs:.5,.5) [cross = .2cm, ultra thick, gray] {};
       \end{axis}
       \end{scope}
%\begin{scope}[xshift=5cm, yshift = 0cm] 
% \begin{axis}[thick, width=4cm, height = 4cm, yticklabels={}, xticklabels={}, axis x line=middle,
%    axis y line=middle,domain = 0:5,
%    axis background/.style={fill=white, opacity = 0.5}, anchor = center, ticks = none, axis line style={draw=gray, opacity = 0.7}, enlargelimits=true]  
%                  {exp(5*x)};
%                   \node at (rel axis cs:.5,.5) [cross = .2cm, ultra thick, gray] {};
%       \end{axis}
%       \end{scope}
      \begin{scope}[xshift=-.7cm, yshift = -2cm] 
\begin{axis}[thick, width=4cm, height = 4cm, yticklabels={}, xticklabels={}, axis x line=middle,
    axis y line=middle,domain = 0:5,
    axis background/.style={fill=white, opacity = 0.5}, anchor = center, ticks = none, axis line style={draw=gray, opacity = 0.7}, enlargelimits=true]
         \addplot[no markers, smooth, domain = 0:5]  
                  {exp(-.7*x)*sin(2*deg(x))};
                   \node at (rel axis cs:.5,.5) [cross = .2cm, ultra thick, gray] {};
       \end{axis}
      \end{scope}
   
    \begin{scope}[xshift=-3cm, yshift = 4cm] 
\begin{axis}[thick, width=4cm, height = 4cm, yticklabels={}, xticklabels={}, axis x line=middle,
    axis y line=middle,domain = 0:5,
    axis background/.style={fill=white, opacity = 0.5}, anchor = center, ticks = none, axis line style={draw=gray, opacity = 0.7}, enlargelimits=true]
         \addplot[no markers, smooth, domain = 0:5]  
                  {exp(-3*x)*sin(4*deg(x))};
                  \node at (rel axis cs:.5,.5) [cross = .2cm, ultra thick, gray] {};
       \end{axis}
      \end{scope}
      
      \begin{scope}[xshift=0cm, yshift = 2cm] 
\begin{axis}[thick, width=4cm, height = 4cm, yticklabels={}, xticklabels={}, axis x line=middle,
    axis y line=middle,domain = 0:5,
    axis background/.style={fill=white, opacity = 0.5}, anchor = center, ticks = none, axis line style={draw=gray, opacity = 0.7}, enlargelimits=true]
         \addplot[no markers, smooth, domain = 0:5]  
                  {sin(2*deg(x))};
                  \node at (rel axis cs:.5,.5) [cross = .2cm, ultra thick, gray] {};
       \end{axis}
      \end{scope}


\end{tikzpicture}
\end{center}
}


%\offslide{Interpretation von Polstellen (zeitdiskrete Systeme)}
\section{Steuerbarkeit}
\frame{\frametitle{Zustandssteuerbarkeit \textit{(state controllability)}}
\framesubtitle{Zustandssteuerbarkeit bewertet, ob alle $x_{i}$ \"uber $u$ beeinflusst werden k\"onnen.}
\begin{definition}[Zustandssteuerbarkeit]
Ein System \eqref{Eq:SSCT1}, \eqref{Eq:SSCT2} bzw. \eqref{Eq:SSDT1},\eqref{Eq:SSDT2} ist dann vollst\"andig zustandssteuerbar, wenn es f\"ur jeden Anfangszustand $x\left(t_{0}\right)$ eine Steuerfunktion $u(t)$ gibt, die das System innerhalb einer endlichen Zeitspanne $t_{0} \leq t \leq t_{1}$ in den Endzustand $x\left(t_{1}\right) = 0 $ \"uberf\"uhrt. 
\end{definition}
Dies gilt genau dann, wenn
\begin{equation}
\rg \left[ B | AB | \cdots |A^{n-1}B \right] = n
\end{equation}

Es wird $\left[ B | AB | \cdots |A^{n-1}B \right]$ die Steuerbarkeitsmatrix genannt.

}

\frame{\frametitle{Ausgangssteuerbarkeit \textit{(controllability)}}
\framesubtitle{Ausgangssteuerbarkeit bewertet, ob  $y$ \"uber $u$ beeinflusst werden k\"onnen.}
\begin{definition}[Ausgangssteuerbarkeit]
Ein System \eqref{Eq:SSCT1}, \eqref{Eq:SSCT2} bzw. \eqref{Eq:SSDT1},\eqref{Eq:SSDT2} ist dann vollst\"andig ausgangssteuerbar, wenn es f\"ur jeden Anfangswert $y\left(t_{0}\right)$ eine Steuerfunktion $u(t)$ gibt, die das System innerhalb einer endlichen Zeitspanne $t_{0} \leq t \leq t_{1}$ in den Endwert $y\left(t_{1}\right)$ \"uberf\"uhrt. 
\end{definition}
Dies gilt f\"ur ein System mit $m$ Ausgangswerten genau dann, wenn
\begin{equation}
\rg \left[ CB | CAB | CA^2 B | \cdots | CA^{n-1}B | D \right] = m
\end{equation}

%Es wird $\left[ B | AB | \cdots |A^{n-1}B \right]$ die Steuerbarkeitsmatrix genannt.

}

%\frame{\frametitle{\"Ubung 3: Steuerbarkeit}
%\framesubtitle{}
%Gegeben sei das System
%\begin{equation}
%\label{Eq:SysCont1}
%\dot{x}(t) = 
%\begin{pmatrix}
% -1 & -1 \\ 1 & -3
%\end{pmatrix} x(t) + 
%\begin{pmatrix}
% 1 \\ 1
%\end{pmatrix} u(t), \quad x(0) = x_{0}
%\end{equation}
%\begin{equation}
%\label{Eq:SysCont2}
%y(t) = x(t)
%\end{equation}
%\begin{block}{\"Ubung 3}
%\"Uberpr\"ufen Sie die Steuerbarkeit und Ausgangssteuerbarkeit des Systems (\ref{Eq:SysCont1}), (\ref{Eq:SysCont2}).
%\end{block}
%}


\frame{\frametitle{Grafische Interpretation von Steuerbarkeit}
\framesubtitle{Zu jedem Systemverhalten der Vergangenheit (Zustand) l\"asst sich ein Funktional $u(t)$ so finden, dass ab $t'$ das gew\"unschte Verhalten herrscht.}
\begin{center}
\begin{picture}(330, 150)(-40,-20)
\setlength{\unitlength}{1.5pt}
\thicklines
\put(0,0){\vector(1, 0){180}}
\put(60,-8){$t$}
\put(0,0){\vector(0, 1){80}}
\put(-9,40){$w$}
\qbezier(5, 5)(20, 5)(30, 20)
\qbezier(30, 20)(40, 35)(60, 30)
\put(10,25){Past}
\qbezier(90, 50)(100, 90)(150, 40)
\put(100,30){Desired}
%\thinlines
{\color{red!80!black}
\qbezier(60, 30)(80, 30)(90, 50)
\put(60,50){Control}}
\end{picture}
\end{center}
}

\subsection{Beobachtbarkeit}
\frame{\frametitle{Beobachtbarkeit \textit{(observability)}}
\framesubtitle{Beobachtbarkeit bewertet, ob alle $x_{i}$ den Ausgang $y$ beeinflussen und somit beobachtet werden k\"onnen.}
\begin{definition}[Beobachtbarkeit]
Ein System \eqref{Eq:SSCT1}, \eqref{Eq:SSCT2} bzw. \eqref{Eq:SSDT1},\eqref{Eq:SSDT2} ist dann vollst\"andig beobachtbar, wenn bei bekannter \"au{\ss}erer Beeinflussung $Bu(t)$ und bekannten Matrizen $A$ und $C$ aus dem Ausgangsvektor $y(t)$ \"uber einem endlichen Zeitintervall $t_{0} \leq t \leq t_{1}$  den Anfangszustand $x\left(t_{0}\right) $ eindeutig bestimmen kann. 
\end{definition}
Dies gilt genau dann, wenn
\begin{equation}
\rg \begin{pmatrix} C \\ CA \\ \vdots \\ C A^{n-1} \end{pmatrix} = n
\end{equation}

Es wird $\begin{pmatrix} C^T & (CA)^T & \cdots & \left(C A^{n-1}\right)^T \end{pmatrix}^T$ die Bobachtbarkeitsmatrix genannt.

}

%\frame{\frametitle{\"Ubung 4: Beobachtbarkeit}
%\framesubtitle{}
%Gegeben sei das System
%\begin{equation}
%\label{Eq:SysCont1}
%\dot{x}(t) = 
%\begin{pmatrix}
% 0 & 1 \\ -1 & -3
%\end{pmatrix} x(t) + 
%\begin{pmatrix}
% 1 \\ 2
%\end{pmatrix} u(t), \quad x(0) = x_{0}
%\end{equation}
%\begin{equation}
%\label{Eq:SysCont2}
%y(t) = 
%\begin{pmatrix}
%1 & 1 
%\end{pmatrix}
%x(t)
%\end{equation}
%\begin{block}{\"Ubung 4}
%\"Uberpr\"ufen Sie die Steuerbarkeit, Ausgangssteuerbarkeit und Beobachtbarkeit des Systems (\ref{Eq:SysCont1}), (\ref{Eq:SysCont2}).
%\end{block}
%}
%

\section{Regelung im Zustandsraum}
%\frame{\sectionpage}
%Zustandsregler
\frame{\frametitle{Regelkreis mit Zustandsr\"uckf\"uhrung}
\framesubtitle{}
         	\begin{center}
            		\begin{tikzpicture}[thick]
  \node[draw,rectangle] (i) {$\int$};
  \node[inner sep=0,minimum size=0,right of=i] (k) {}; 
  \node[inner sep=0,minimum size=0,right of=k] (m) {}; 
  \node[draw, circle, left of=i] (l){};
  \node[draw, rectangle, left of=l] (b){$B$};
  \node[draw, circle, left of=b] (n) {}; 
  \node[draw,rectangle,right of=m] (c) {$C$};
  \node[draw,rectangle,below of=i] (a) {$A$};
  \node[draw,rectangle,below of=a] (f) {$F$};
  \node[inner sep=0,minimum size=0,left of=n] (o) {};
  \node[inner sep=0,minimum size=0,right of=c] (p) {}; 
  \node[inner sep=0,minimum size=0,right of=c] (p) {};
  %\node[<(<name>) at (<coordinate>){<text>};
  

  % 1st pass: draw arrows
  \draw[vecArrow] (i) -- (c) node[near start, above] {$x$};
  \draw[vecArrow] (c) -- (p) node[midway, above] {$y$};
  \draw[vecArrow] (k) |- (a);
  \draw[vecArrow] (a) -| (l);
  \draw[vecArrow] (l) -- (i) node[midway, above] {$\dot{x}$};
  \draw[vecArrow] (b) to (l);
  \draw[vecArrow] (m) |- (f);
  \draw[-stealth] (f) -| (n) node[pos = .95, right] {$-$};
  \draw[vecArrow] (n) to (b);
  \draw[vecArrow] (o) -- (n) node[midway, above] {$u$};
  

  % 2nd pass: copy all from 1st pass, and replace vecArrow with innerWhite
 \draw[innerWhite] (i) -- (c);
  \draw[innerWhite] (c) -- (p);
  \draw[innerWhite] (k) |- (a);
  \draw[innerWhite] (a) -| (l);
  \draw[innerWhite] (l) -- (i);
  \draw[innerWhite] (b) to (l);
  \draw[innerWhite] (m) |- (f);
  %\draw[innerWhite] (f) -| (n);
  \draw[innerWhite] (n) to (b);
  \draw[innerWhite] (o) -- (n); 
  % Note: If you have no branches, the 2nd pass is not needed
\end{tikzpicture}
\end{center}
Zustandsraumdarstellung f\"ur $D = 0$:
 \begin{eqnarray}
\frac{\D x}{\D t} &=& \left(A-BF\right)x+Bu \\
y &=& C x
\end{eqnarray}
  
}

%\section{Regelung im Zustandsraum}
%Zustandsregler
\frame{\frametitle{Regelkreis mit Ausgangsr\"uckf\"uhrung}
\framesubtitle{}
         	\begin{center}
            		\begin{tikzpicture}[thick]
  \node[draw,rectangle] (i) {$\int$};
  \node[inner sep=0,minimum size=0,right of=i] (k) {}; 
  \node[inner sep=0,minimum size=0,right of=k] (m) {}; 
  \node[draw, circle, left of=i] (l){};
  \node[draw, rectangle, left of=l] (b){$B$};
  \node[draw, circle, left of=b] (n) {}; 
  \node[draw,rectangle,right of=m] (c) {$C$};
  \node[draw,rectangle,below of=i] (a) {$A$};
  \node[draw,rectangle,below of=a] (f) {$F'$};
  \node[inner sep=0,minimum size=0,left of=n] (o) {};
  \node[inner sep=0,minimum size=0,right of=c] (p) {}; 
  \node[inner sep=0,minimum size=0,right of=p] (q) {};
  %\node[<(<name>) at (<coordinate>){<text>};
  

  % 1st pass: draw arrows
  \draw[vecArrow] (i) -- (c) node[near start, above] {$x$};
  \draw[-stealth] (c) -- (q) node[midway, above] {$y$};
  \draw[vecArrow] (k) |- (a);
  \draw[vecArrow] (a) -| (l);
  \draw[vecArrow] (l) -- (i) node[midway, above] {$\dot{x}$};
  \draw[vecArrow] (b) to (l);
  \draw[] (p) |- (f);
  \draw[-stealth] (f) -| (n) node[pos = .95, right] {$-$};
  \draw[vecArrow] (n) to (b);
  \draw[vecArrow] (o) -- (n) node[midway, above] {$u$};
  

  % 2nd pass: copy all from 1st pass, and replace vecArrow with innerWhite
 \draw[innerWhite] (i) -- (c);
 % \draw[innerWhite] (c) -- (q);
  \draw[innerWhite] (k) |- (a);
  \draw[innerWhite] (a) -| (l);
  \draw[innerWhite] (l) -- (i);
  \draw[innerWhite] (b) to (l);
  %\draw[innerWhite] (p) |- (f);
  %\draw[innerWhite] (f) -| (n);
  \draw[innerWhite] (n) to (b);
  \draw[innerWhite] (o) -- (n); 
  % Note: If you have no branches, the 2nd pass is not needed
\end{tikzpicture}
\end{center}
Zustandsraumdarstellung f\"ur $D = 0$:
 \begin{eqnarray}
\frac{\D x}{\D t} &=& \left(A-BF'C \right)x+Bu \\
y &=& C x
\end{eqnarray}
  
}


\frame{\frametitle{Vor- und Nachteile Regelung im Zustandsraum}
\framesubtitle{}
\begin{itemize}
\item Vorteile
\begin{itemize}
		\item (Fast) vollst\"andige Systembeeinflussung: \\
		\begin{itemize}
		\item $A$ \"uberf\"uhrt in $ \left(A-BF\right)$ bzw. $\left(A-BF'C \right)$
		\end{itemize}
		\item Weitestgehend algebraische Rechenoperationen
		\item Einfache Implementierung im Controller
		\item Intuitive Modellierung des Energiegehalts
		\end{itemize} 
\item Nachteile
\begin{itemize}
		\item Reglervorgabe (Eigenwerte) un\"ublich
		\item Zustandsmessung oder Beobachter n\"otig
		\item Genaue Kenntnis der Systemparameter notwendig
		\end{itemize}
\end{itemize}
}
%\section{Beobachter}
%\subsection{Einf\"uhrung}
%\frame{\frametitle{Sch\"atzfunktionen}
%\framesubtitle{Eine Sch\"atzfunktion (Sch\"atzer) dient zur Ermittlung eines Parameter-Sch\"atzwertes bzw. zur Ermittlung eines wahrscheinlichen rauschfreien Zustands aus empirischen Daten }
%\begin{columns}[t] 
%     \begin{column}[T]{5cm} 
%     	\begin{itemize}
%     		\item Grundlage: endlich viele Beobachtungen (Stichprobe)
%		\begin{itemize}
%			\item Sch\"atzer selbst fehlerbehaftet
%			\item H\"aufig Zufallsvariable
%		\end{itemize}
%		\item Schlu{\ss} auf Grundgesamtheit
%		\item Sch\"atzen einzelner Parameter der Verteilung
%		\begin{itemize}
%			\item Mittelwert
%			\item Median
%			\item Standardabweichung
%		\end{itemize}
%     	\end{itemize}
%     \end{column}
%     	\begin{column}[T]{6cm} 
%		\begin{definition}[Zufallsvariable]
%		Als Zufallsvariable bezeichnet man eine messbare Funktion von einem Wahrscheinlichkeitsraum in einen Messraum.
%		\end{definition}
%		\begin{definition}[Sch\"atzfunktion]
%		Eine Sch\"atzfunktion dient dazu, aufgrund von empirischen Daten einer Stichprobe einen Schätzwert zu ermitteln und dadurch Informationen über unbekannte Parameter einer Grundgesamtheit zu erhalten.
%		\end{definition}
%     \end{column}
% \end{columns}
%}
%
%
%\frame{\frametitle{Sch\"atzfunktionen und Eigenschaften}
%\framesubtitle{G\"angige Sch\"atzfunktionen und w\"unschenswerte Eigenschaften}
%\begin{columns}[t] 
%     \begin{column}[T]{6cm} 
%     	\begin{itemize}
%     		\item Mittelwert 
%		\begin{eqnarray*}
%		\bar{X} = \frac{1}{n} \sum_{i=1}^{n} X_{i} \\
%		\hat{\mu} = \bar{x} = \frac{1}{n} \sum_{i=1}^{n} x_{i} 
%		\end{eqnarray*}
%		\item Varianz 
%		\begin{eqnarray*}
%		S_{n}^{2} = \frac{1}{n-1} \sum_{i=1}^{n}\left(X_{i} - \bar{X}\right)^{2} \\
%		\hat{\sigma}^{2} = s_{n}^{2} = \frac{1}{n-1} \sum_{i=1}^{n}\left(x_{i} - \bar{x}\right)^{2}
%		\end{eqnarray*}
%     	\end{itemize}
%     \end{column}
%     	\begin{column}[T]{5cm} 
%         	\begin{itemize}
%     		\item Erwartungstreue:
%		\begin{itemize}
%		\item Erwartungswert der Sch\"atzfunktion gleich wahrem Parameter
%		\item Kein systematischer Fehler (Bias).
%		\end{itemize}
%		\item Konsistenz:
%		\begin{itemize}
%		\item Unsicherheit des Sch\"atzers nimmt f\"ur $n \rightarrow \infty$ ab
%		\end{itemize}
%		\item Effizienz:
%		\begin{itemize}
%		\item Minimale Varianz des Sch\"atzers
%		\end{itemize}
%		\item BLUE: Best Linear Unbiased Estimator (auch: optimales Filter)
%     	\end{itemize}
%     \end{column}
% \end{columns}
%}
%
%\frame{\frametitle{AutoRegressive Model with eXogenous inputs (ARX)}
%\framesubtitle{F\"ur Parametersch\"atzer h\"aufig eingesetzte Modellstruktur.}
%Das Modell eines Eingr\"o{\ss}ensystems sei beschrieben durch
%\begin{equation}
%y_{k} + a_{1}y_{k-1} +  \cdots + a_{n_{a}} y_{k-n_{a}}= b_{1} u_{k-1} + \cdots + b_{n_{b}} u_{k-n_{b}} + e_{k}
%\end{equation}
%mit 
%\begin{itemize}
%		\item $n_{b} \leq n_{a}$, 
%		\item Parametervektor $\theta = \begin{pmatrix}a_{1} &a_{2} &\cdots &a_{n_{a}} &b_{1}& b_{2}& \cdots& b_{n_{b}}\end{pmatrix}$, 
%		\item Eingangswert $u_{k}$,
%		\item Ausgangswert $y_{k}$ sowie 
%		\item additivem wei{\ss}en Rauschen $e_{k}$
%		\end{itemize}
%}
%
%\frame{\frametitle{Beobachtungsmatrix \textit{(Observation matrix)}}
%\framesubtitle{Die Beobachtungmatrix $H$ sammelt $N$ Beobachtungen eines ARX Systems.}
%\begin{equation}
%\begin{split}
%		H = \begin{pmatrix} 
%		-y_{k-1} & \cdots & -y_{k-n_{a}} & u_{k-1} & \cdots & u_{k-n_{b}} \\
%		-y_{k-2} & \cdots & -y_{k-n_{a}-1} & u_{k-2} & \cdots & u_{k-n_{b}-1}\\
%		\vdots & \cdots & \vdots & \vdots & \cdots & \vdots \\
%		-y_{\left(k-N-1\right)} & \cdots & -y_{\left(k-n_{a}-N\right)} & u_{\left(k-N-1\right)} & \cdots & u_{\left(k-n_{b}-N\right)} \end{pmatrix} \\
%		\in \mathbb{R}^{N \times \left(n_{a}+n_{b}\right)}
%		\end{split}
%		\end{equation}
%		
%Mit einem Bobachtungsvektor $h = \begin{pmatrix} 
%		-y_{k-1} & \cdots & -y_{k-n_{a}} & u_{k-1} & \cdots & u_{k-n_{b}} \end{pmatrix}$ gilt:
%		\begin{equation*}
%		y_{k} = h \theta
%		\end{equation*}
% 
%}
%
%\frame{\frametitle{Formulierung des Optimierungsziels}
%\framesubtitle{Das Optimierungsziel f\"ur Parametersch\"atzer wird \"uber die Summe der Fehlerquadrate des Ausgangs definiert.}
%Sei die Summe der Fehlerquadrate \textit{(Sum of Square Errors, SSE)} f\"ur einen Parametervektor $\theta$ und eine Indexmenge $I \subset \mathbb{N}$ definiert als
%\begin{equation}
%E\left( \theta \right) = \sum_{k \in I} \left( y_{k} - \hat{y}_{k|\theta} \right)^2
%\end{equation}
%Hierbei beschreibt $\hat{y}_{k|\theta}$ den gesch\"atzten Ausgangswert f\"ur einen Parametervektor $\theta$.
%
%Das Optimierungsziel ist damit
%\begin{equation}
%\label{Eqn-ThetaHat}
%\hat{\theta} = %\hat{\theta}(\mathbf{Z}) = \mathrm{arg\, min}_\theta E(\theta, \mathbf{Z}) =
% \mathrm{arg\, min}_\theta  \sum_{k \in I} \left(y_{k} - \hat{y}_{k|\theta} \right)^2
%\end{equation}
%}
%
%
%\frame{\frametitle{Lineare Rekursion \textit{(Linear Least Squares)}}
%\framesubtitle{}
%\begin{columns}[t] 
%     \begin{column}[T]{5cm} 
%     	\begin{itemize}
%     		\item H\"aufig eingesetzt
%		\item Kein Online-Sch\"atzer
%		\item In vielen Software-Produkten implementiert
%		\item Optimal im Sinne des Optimierungsziels
%		\item Stabil
%		\item Variante: Block Least Squares
%	\end{itemize}
%     \end{column}
%     	\begin{column}[T]{7cm} 
%            Formuliere \eqref{Eqn-ThetaHat} als
%		\begin{equation}
%			E(\theta) = \left(\Id - H\theta\right)^T \left(\Id - H\theta\right)
%		\end{equation}
%	        Damit ist
%	        \begin{equation}
%	        \begin{split}
%		\frac{\D E(\theta)}{\D \theta} %= -2H^T \Id+ \left(X^T X \right)^T \theta + \left(X^T X \right) \theta \\
%		= -2H^T \Id + 2 H^T H \theta
%		\end{split}
%		\end{equation}
%		und der Least Squares Estimator f\"ur $\theta$, bezeichnet als $\hat{\theta}$, ist
%		\begin{equation}
%			\hat{\theta} = \left(H^T H\right)^{-1} H^T \Id
%		\end{equation}
%		Hierbei ist $\left(H^T H\right)^{-1} H^T$ die Moore-Penrose Pseudoinverse einer nichtquadratischen Matrix.
%     \end{column}
% \end{columns}
%}
%
%\frame{\frametitle{Recursive Least Squares}
%\framesubtitle{Idee: Rekursives Update der Beobachtungsmatrix bzw. der Moore-Penrose Pseudoinversen}
%\begin{lemma}[Matrix Inversion]
%\label{Lemma-MatrixInversion}
%Let $\mathbf{\mathbf{A}}$, $\mathbf{\mathbf{C}}$ and $\mathbf{A} + \mathbf{B}\mathbf{C}\mathbf{D}$ be nonsingular square matrices, then the following identity holds:
%\begin{equation}
%\label{eq-MatrixInversionLemma}
%(\mathbf{A} + \mathbf{B}\mathbf{C}\mathbf{D})^{-1} = \mathbf{A}^{-1} - \mathbf{A}^{-1} \mathbf{B} \left(\mathbf{C}^-1 + \mathbf{D} \mathbf{A}^{-1} \mathbf{B} \right)^{-1}\mathbf{D}\mathbf{A}^{-1}.
%\end{equation}
%\end{lemma}
%
%Aktualisierte Bebachtungsmatrix:
%\begin{equation}
%\label{Eqn-UpdatedObsMatrix}
%H_{n+1} = \begin{pmatrix}
%      H_{n}    \\ \hline
%      h_{n+1} 
%\end{pmatrix}
%\end{equation}
%Aktualisierter Ausgangsvektor:
%\begin{equation}
%\label{Eqn-UpdatedImColumn}
%\mathbf{y}_{n+1} = \begin{pmatrix}
%      \mathbf{y}_{n}    \\ \hline
%      y_{n+1} 
%\end{pmatrix}
%\end{equation}
%}
%
%\frame{\frametitle{RLS ohne forgetting factor}
%\framesubtitle{}
%Es ist m\"oglich, $\hat{\theta}$ zu schreiben als 
%\begin{equation}
%\label{Eqn-RLSLeastSquaresEstimate}
%\hat{\theta}_{n+1} = \Phi_{n+1}H_{n+1}^T \mathbf{y}_{n+1},
%\end{equation}
%wobei $\hat{\theta}_{n+1}$ \emph{gesch\"atzter Parametervektor f\"ur Daten einschlie{\ss}lich $n+1$}  und $ \Phi = \left(H^T H \right)^{-1}$ \footnote{error covariance matrix} bedeutet. Es gilt 
%\begin{equation}
%\label{Eqn-ErrorCovMatrixPartitioned}
%\begin{split}
%\Phi_{n+1}& = \left(\begin{pmatrix}
%      H_{n}    \\ \hline
%      h^T_{n+1} 
%\end{pmatrix}^T \begin{pmatrix}
%      H_{n}    \\ \hline
%      h^T_{n+1} 
%\end{pmatrix} \right) ^{-1} 
%= \left( H_n^T H_n + h_{n+1}h_{n+1}^T \right)^{-1} \\
%& = \left( \Phi^{-1}_n + h_{n+1} h_{n+1}^T\right)^{-1}.
%\end{split}
%\end{equation}
%Mit dem Matrix Inversion Lemma:
%\begin{equation}
%\label{Eqn-CovarianceMatrixUpdated}
%\Phi_{n+1} = \Phi_n - \Phi_n h_{n+1}  \left(\Id + h_{n+1}^T \Phi_n h_{n+1} \right)^{-1} h_{n+1}^T \Phi_n.
%\end{equation}
%Da $\left(\Id + h_{n+1}^T \Phi_n h_{n+1} \right)$ ein Skalar ist, gilt:
%\begin{equation}
%\label{Eqn-CovarianceMatrixUpdatedCommon}
%\Phi_{n+1} = \Phi_n - \frac{\Phi_n h_{n+1}   h^T_{n+1} \Phi_n}{1 + h^T_{n+1}\Phi_n h_{n+1} }
%\end{equation}
%}
%
%\frame{\frametitle{RLS Algorithmus}
%\framesubtitle{}
%\begin{eqnarray}
% \phi_n & = & \Phi_{n-1} h_n \left( 1 + h^T_n \Phi_{n-1} h_n \right) \\
%\hat{\theta}_n & = & \hat{\theta}_{n-1} + \phi_n h_{n} \left(i_n - h^T_{n} \hat{\theta}_{n-1} \right) \\
%\label{Eqn-ForgettingFactorsHere}
%\Phi_n & = & \left(\Id - \phi_n h^T_n \right) \Phi_{n-1}, 
%\end{eqnarray}
%Initialisierung:
%\begin{itemize}
%		\item Anfangswerte $\Phi_0$ and $\hat{\theta}_0$ gem\"a{\ss} \textit{a priori} Wissen
%		 \begin{itemize}
%		\item Kein \textit{a priori} Wissen: $\hat{\theta}_0 = \mathbf{0}$ und $\Phi_0 = 10^3 \Id$
%		\item \textit{a priori} Wissen vorhanden: $\hat{\theta}_{0}$ auf bekannte Werte, $\Phi_{0}$ reduzieren
%		\end{itemize}
%\item Kovarianz Matrix $\Phi$ ist f\"ur normalverteiltes wei{\ss}es Rauschen mit Standardabweichung $\sigma$ proportional zur Fehlerkovarianzmatrix
%\begin{equation}
%\label{Eqn-EstimationErrorCovMatrix}
%\Phi_n = \frac{1}{\sigma^2} \mathrm{cov} \left(\hat{\theta}_n - \theta_n \right) = 
%\frac{1}{\sigma^2} \mathrm{E}  \left( \left(\hat{\theta}_n - \theta_n \right) \left(\hat{\theta}_n - \theta_n \right)^T \right).
%\end{equation}
%\end{itemize}}
%
%\frame{\frametitle{RLS Algorithmus: Probleme}
%\framesubtitle{}
%\begin{itemize}
%\item Covariance Blow Up:
%\begin{itemize}
%		\item Verursacht durch fast linear abh\"angige Beobachtungen
%		\item Abhilfe: Kleine St\"orung hinzuf\"ugen (An der Startbasis r\"utteln!)
%		\item Alternativ: Forgetting factors
%		\end{itemize}
%\item Numerische Instabilit\"at:
%\begin{itemize}
%		\item Die Eintr\"age in $\Phi$ werden klein f\"ur eingeschwungene Systeme
%		\item $\Phi$ muss positive definit sein
%		\item Abhilfe: Kleine St\"orung hinzuf\"ugen (An der Startbasis r\"utteln!)
%		\end{itemize}
%\end{itemize}
%}
%
%\frame{\frametitle{RLS Algorithmus: Forgetting Factors etc.}
%\framesubtitle{Ohne Ma{\ss}nahmen konvergiert der RLS gegen das Ergebnis des LLS.}
%\begin{itemize}
%\item Standard-RLS: Konvergenz gegen LLS-Sch\"atzung
%	\begin{itemize}
%		\item Keine optimale Sch\"atzung f\"ur zeitabh\"angige Systeme
%	\end{itemize}
%\item Abhilfe: 
%\begin{itemize}
%		\item Forgetting factor: Koeffizient zur exponentiellen Abwertung der Daten aus der Vergangenheit
%		\begin{itemize}
%		\item Fixed forgetting factor: $\lambda < 1$, \"ublich: $\lambda = 0.95\ldots 0.99$
%		\item Faustregel: RLS mit fixed forgetting betrachtet $M = \frac{1}{1-\lambda}$ Beobachtungen $h_{i}$
%		\item Variable forgetting: Verschiedene Verfahren in der Regel abh\"angig von der Prediktionsg\"ute
%		\end{itemize}
%		\item Covariance reset: $\Phi$ wird bei Vorliegen gewisser Kriterien zur\"uckgesetzt
%		\begin{itemize}
%		\item F\"ur abrupte System-Ver\"anderungen
%		\item R\"ucksetzwert bestimmt St\"arke der Wirkung
%		\end{itemize}
%		\end{itemize}
%\end{itemize}
%}
%
%\frame{\frametitle{RLS mit forgetting factor}
%\framesubtitle{}
%\begin{itemize}
%\item Cost function angepasst:
%\begin{equation}
%\label{Eqn-RLSFFNorm}
%J_N(\theta) = \left(\Id - H\theta\right)^T
%\begin{pmatrix}
%  \lambda^N    & \cdots & 0  & 0 \\
%    \vdots  & \ddots & \vdots & \vdots \\
%    0 & \cdots & \lambda^2 & 0 \\
%    0 & \cdots & 0 & \lambda
%\end{pmatrix}
%\left(\Id - H\theta\right)
%\end{equation}
%\item M\"oglich durch \"Anderung von \eqref{Eqn-ForgettingFactorsHere} in
%\begin{equation}
%\label{Eqn-CovUpdateFF}
%\Phi_n = \frac{1}{\lambda}\left(\mathbf{1} - \phi_n \mathbf{x}^T_n \right)\Phi_{n-1}.
%\end{equation}
%\item Mit abnehmendem $M$ steigt die Empfindlichkeit f\"ur Rauschen
%\end{itemize}
%}
%
%
%\frame{\frametitle{Kalman Filter - Grundlagen}
%\framesubtitle{}
%\begin{itemize}
%\item Grundlage: Diskrete Zustandsraumbeschreibung
%\begin{eqnarray}
%\label{Eqn-SSDynSys1}
% {x}_{k+1} & = &  {A}  {x}_k +  {B}  {u}_k +  {E}  {w}_k\\
%\label{Eqn-SSDynSys2}
%y_{k+1} & = &  C  {x}_{k+1} + r_{k+1} 
%\end{eqnarray}
%\item Process Noise Vector $w_{k}$, Output Noise Vector $r_{k+1}$
%\begin{itemize}
%		\item Wei{\ss}, d.h. $\E\left(w_{k}\right) = \E\left(r_{k+1}\right) = 0$
%		\item Wechselseitig unkorreliert
%		\item Bekannte Varianz $\Var\left(w\right) = R_{w}$ bzw. $\Var\left(r\right) = r_{v}$
%		\end{itemize}
%\item Systemparameter konstant und bekannt
%\end{itemize}
%}
%
%\frame[allowframebreaks]{\frametitle{Kalman Filter}
%\framesubtitle{}
%\begin{itemize}
%\item Ziel des Kalman Filters:
%\begin{itemize}
%		\item Vorhersage f\"ur $x_{k}$
%		\item Minimierung des Vorhersagefehlers $\hat{x}_k - x_k$ nur mittels Messung des Eingangs $u_{k}$ und des Ausgangs $y_{k}$ 
%		\end{itemize}
%\item Beste Vorhersage f\"ur bekanntes $x_{k-1}$ ist der rauschfreie Zustand (wg. $\E\left(w_{k}\right) = \E\left(r_{k+1}\right) = 0$)
%\begin{equation}
%\label{Eqn-KFNoiseFreeEst}
%\hat{{x}}_{k|k-1} = {A} \hat{{x}}_{k-1} + {B} {u}_{k-1},
%\end{equation}
%\item Der Vorhersagefehler wird damit
%\begin{equation}
%\label{Eqn-KFPredError1}
%{e}_{k|k-1} = \hat{{x}}_{k|k-1} - {x}_k,
%\end{equation}
%\item Aufgrund der Linearit\"at des Systems und der Wahl des Sch\"atzers
%\begin{equation}
%\begin{split}
%\label{Eqn-KFPredError2}
% {e}_{k|k-1} &=  {A} \hat{ {x}}_{k-1} +  {B u}_{k-1} -  {A x}_{k-1} -  {B u}_{k-1} -  {D w}_{k-1} \\
%& =  {A} \left( \hat{ {x}}_{k-1} -  {x}_{k-1} \right) -  {D w}_{k-1}.
%\end{split}
%\end{equation}
%\item Die Fehlerkovarianzmatrix wird damit
%\begin{equation}
%\label{Eqn-KFSECovMat}
%\Phi_{k|k-1} = \mathrm{E} \left\{{e}_{k|k-1} {e}_{k|k-1}^T \right\}
%\end{equation}
%\item mit
%\begin{equation}
%\label{Eqn-ErrorCov}
%\begin{split}
% &{e}_{k|k-1}  {e}_{k|k-1}^T \\
% &= \left(  {A} \left( \hat{ {x}}_{k-1} -  {x}_{k-1} \right) -  {E w}_{k-1} \right)
%\left(  \left( \hat{ {x}}_{k-1} -  {x}_{k-1} \right)^T  {A}^T -  {w}_{k-1}^T  {E}^T \right) \\
%& =  {A} \left( \hat{ {x}}_{k-1} -  {x}_{k-1} \right) 
%\left( \hat{ {x}}_{k-1} -  {x}_{k-1} \right)^T  {A}^T
%-   {A} \left( \hat{ {x}}_{k-1} -  {x}_{k-1} \right)  {w}_{k-1}^T  {E}^T \\
%&+  {E w}_{k-1} \left( \hat{ {x}}_{k-1} -  {x}_{k-1} \right)^T  {A}^T +  {E w}_{k-1}  {w}_{k-1}^T  {E}^T
%\end{split}
%\end{equation}
%\item Auf Grund der Linearit\"at des Erwartungswertes k\"onnen die Erwartungswerte der einzelnen Terme berechnet werden
%\begin{eqnarray}
%\mathrm{E} \left\{ {A} \left( \hat{{x}}_{k-1} - {x}_{k-1} \right) 
%\left( \hat{{x}}_{k-1} - {x}_{k-1} \right)^T {A}^T \right\} & = & {A} \Phi_{k-1} {A}^T \\
%\mathrm{E} \left\{ {A} \left( \hat{{x}}_{k-1} - {x}_{k-1} \right) {w}_{k-1}^T {E}^T \right\} & = & 0 \\
%\mathrm{E} \left\{ {E w}_{k-1} \left( \hat{{x}}_{k-1} - {x}_{k-1} \right)^T {A}^T \right\} & = & 0 \\
%\mathrm{E} \left\{ {E w}_{k-1} {w}_{k-1}^T {E}^T \right\} & = & {E} {R}_w {E}^T,
%\end{eqnarray}
%\item damit folgt
%\begin{equation}
%\label{Eqn-KFSECovMatFinal}
%\Phi_{k|k-1} =  {A} \Phi_{n-1} {A}^T  + {E} {R}_w {E}^T
%\end{equation}
%\end{itemize}
%}
%
%\frame{\frametitle{Kalman Fiter als Rekursion}
%\framesubtitle{}
%\begin{itemize}
%\item Vorhersage \textit{(prediction stage)}
%\begin{eqnarray}
%\hat{{x}}_{k|k-1} &=& {A} \hat{{x}}_{k-1} + {B} {u}_{k-1} \\
%\Phi_{k|k-1} &= & {A} \Phi_{k-1} {A}^T  + {D} {R}_w {D}^T 
%\end{eqnarray}
%\item Korrektur \textit{(correction stage)}
%\begin{eqnarray}
%\phi_k & = & \Phi_{k|k-1} C^T \left(r_v + C_k \Phi_{k|k-1} C_{k}^T \right)^{-1} \\
%\hat{{x}}_k & = & \hat{{x}}_{k|k-1} + \phi_k  \left(y_k - C_{k} \hat{{x}}_{k|k-1} \right) \\
%\Phi_k & = & \left(\Id - \phi_k C_k \right) \Phi_{k|k-1}.
%\end{eqnarray}
%\end{itemize}
%}
%
%\frame{\frametitle{Kalman Filter f\"ur Parametersch\"atzung}
%\framesubtitle{}
%\begin{itemize}
%\item Durch \"Uberf\"uhrung $ {x} \rightarrow \theta$, $C^T \rightarrow {x}$ kann das Kalman Filter f\"ur Parametersch\"atzung genutzt werden:
%\item Vorhersage
%\begin{eqnarray}
%\hat{\theta}_{k|k-1} &=& \hat{\theta}_{k-1} \\
%\Phi_{k|k-1} &= &\Phi_{k-1} +  {R}_w
%\end{eqnarray}
%\item Korrektur
%\begin{eqnarray}
%\phi_k & = & \Phi_{k|k-1} {x}_k \left(r_v + {x}^T_k \Phi_{k|k-1} {x}_k \right)^{-1} \\
%\hat{\theta}_k & = & \hat{\theta}_{k|k-1} + \phi_k  \left(y_k - {x}_{k}^T \hat{\theta}_{k|k-1} \right) \\
%\Phi_k & = & \left({1} - \phi_k {x}^T_k \right) \Phi_{k|k-1}.
%\end{eqnarray}
%
%\end{itemize}
%}
%
%\frame{\frametitle{Kalman Filter: Initialisierung}
%\framesubtitle{\"Ubliche Werte f\"ur die Initialisierung eines KF in der Praxis}
%\begin{itemize}
%\item $R_{w} = \begin{pmatrix} \sigma_{1}^2 &\cdots &0 \\
%\vdots & \ddots & \vdots \\
%0 & \cdots & \sigma_{n}^2 \end{pmatrix}$
%\item $x =x_{\mathrm{true}}$ oder $x = 0$
%\item $\Phi = 10^4 \Id$
%\item $r_{v} = 1$
%\end{itemize}
%}
%
%\frame{\frametitle{Kalman Filter: Erweiterungen}
%\framesubtitle{}
%\begin{itemize}
%\item Extended Kalman Filter (EKF):
%\begin{itemize}
%		\item Sch\"atzung von Parametern und Zustand
%		\item Ber\"ucksichtigung von Nichtlinearit\"aten
%\end{itemize}
%\item Errors-In-Variables Kalman Filter (EIV-KF):
%\begin{itemize}
%		\item Orthogonale Sch\"atzung (Fehler in beiden Variablen)
%\end{itemize}
%\item Errors-In-Variables Kalman Filter (EIV-EKF):
%\begin{itemize}
%		\item Orthogonale Sch\"atzung (Fehler in beiden Variablen)
%		\item Sch\"atzung von Parametern und Zustand
%		\item Ber\"ucksichtigung von Nichtlinearit\"aten		
%\end{itemize}
%\item Unscented Kalman Filter (UKF):
%\begin{itemize}
%		\item Statistische Ber\"ucksichtigung von Nichtlinearit\"aten
%		\end{itemize}		
%\end{itemize}
%}
%
%\offslide{Luenberger Beobachter: Struktur}
%
%
%\frame{\frametitle{Luenberger-Beobachter: Zustandsraumbeschreibung}
%\framesubtitle{}
%\begin{itemize}
%\item Zustandsgleichung des Beobachters:
%\begin{equation}
%\dot{x} = A \hat{x} + B u + H \left(y - \hat{y}\right)
%\end{equation}
%\item Ausgangswert des Beobachters
%\begin{equation}
%\hat{y} = C \hat{x}
%\end{equation}
%\item Damit folgt f\"ur den Sch\"atzwert des Zustands
%\begin{equation}
%\hat{x} = A \hat{x} + B u + H C \left(x- \hat{x}\right)
%\end{equation}
%\item Der Sch\"atzfehler $e = x- \hat{x}$ gen\"ugt damit der Zustandsgleichung 
%\begin{equation}
%\dot{e} = \left(A - H C \right) e
%\end{equation}
%\item Ist $\left(A - H C \right)$ stabil, gilt $\lim_{t \rightarrow \infty} e(t) = 0$
%\item Eigenwerte von $\left(A - H C \right)$ sollten links von den Eigenwerten von $A$ liegen, damit ist der Beobachter schneller als das System
%\end{itemize}
%}


\begin{frame}[allowframebreaks]
\frametitle{Literatur}
\framesubtitle{}
\nocite{unbehauen2}
\nocite{Franklin}
\nocite{ljung99sysid}
\nocite{lunze2010regelungstechnik,lunze2010regelungstechnik2}
\bibliographystyle{abbrv}
\bibliography{../../../bib}
\end{frame}
%
%
%
%\frame{\frametitle{Proof Matrix Inversion Lemma}
%\framesubtitle{}
%\textbf{Proof} The proof follows \cite{hsia}. Pre-multiplying  \eqref{eq-MatrixInversionLemma} by $\mathbf{A} + \mathbf{B}\mathbf{C}\mathbf{D}$ results in
%\begin{equation}
%\label{Eq-ProofMatrixInvLemma}
%\mathbf{1} = \left(\mathbf{A} + \mathbf{B}\mathbf{C}\mathbf{D} \right) \left(\mathbf{A}^{-1} - \mathbf{A}^{-1}\mathbf{B} \left(\mathbf{C}^{-1} + \mathbf{D}\mathbf{A}^{-1}\mathbf{B}\right)^{-1} \mathbf{D} \mathbf{A}^{-1} \right)
%\end{equation}
%and the objective of the proof is to show that the right hand side of \eqref{Eq-ProofMatrixInvLemma} is the identity. By direct manipulation it is possible to obtain
%\begin{displaymath}
%\label{Eqn-ProofMatrixInversionLemma2}
%\begin{split}
%& \left(\mathbf{A} + \mathbf{B}\mathbf{C}\mathbf{D} \right) \left(\mathbf{A}^{-1} - \mathbf{A}^{-1}\mathbf{B} \left(\mathbf{C}^{-1} + \mathbf{D}\mathbf{A}^{-1}\mathbf{B}\right)^{-1} \mathbf{D} \mathbf{A}^{-1} \right) \\
% = & \mathbf{1} + \mathbf{B}\mathbf{C}\mathbf{D}\mathbf{A}^{-1} - \mathbf{B} \left(\mathbf{C}^{-1} + \mathbf{D} \mathbf{A}^{-1} \mathbf{B} \right)^{-1} \mathbf{D} \mathbf{A}^{-1} \\
% & - \mathbf{B}\mathbf{C}\mathbf{D}\mathbf{A}^{-1}\mathbf{B}\left(\mathbf{C}^{-1} + \mathbf{D} \mathbf{A}^{-1}\mathbf{B} \right)^{-1} \mathbf{D} \mathbf{A}^{-1} \\
% = & \mathbf{1} + \mathbf{B}\mathbf{C}\mathbf{D}\mathbf{A}^{-1} - \mathbf{B}\left(\mathbf{1} + \mathbf{C}\mathbf{D}\mathbf{A}^{-1}\mathbf{B}\right)\left(\mathbf{C}^{-1} + \mathbf{D} \mathbf{A}^{-1} \mathbf{B} \right)^{-1}\mathbf{D} \mathbf{A}^{-1}\\
% = & \mathbf{1} + \mathbf{B}\mathbf{C}\mathbf{D}\mathbf{A}^{-1} - \mathbf{B}\mathbf{C}\left(\mathbf{C}^{-1} + \mathbf{D}\mathbf{A}^{-1}\mathbf{B}\right)\left(\mathbf{C}^{-1} + \mathbf{D}\mathbf{A}^{-1}\mathbf{B}\right)^{-1}\mathbf{D}\mathbf{A}^{-1} \\
% = & \mathbf{1} + \mathbf{B}\mathbf{C}\mathbf{D}\mathbf{A}^{-1} - \mathbf{B}\mathbf{C}\mathbf{D}\mathbf{A}^{-1}\\
% = & \mathbf{1}
%\end{split}
%\end{displaymath}
%This proofs the Matrix Inversion Lemma.
%}


\end{document}