\documentclass[11pt,a4paper,headsepline]{scrartcl}
\usepackage[utf8]{inputenc}
\usepackage[T1]{fontenc}
\usepackage[ngerman]{babel}
\usepackage{amsmath}
\usepackage{amsthm}
\usepackage{amssymb}
\usepackage{amsfonts}
\usepackage[scaled]{helvet}
\usepackage{amssymb}
\usepackage{multirow}
\usepackage{textcomp}
\usepackage{graphicx}
\usepackage{paralist}
\usepackage{textcomp}
\usepackage{pdflscape} 
\usepackage{marvosym}
\usepackage{float}
\usepackage{siunitx}
\usepackage[europeanresistors,europeaninductors]{circuitikz}
\usepackage{fancyhdr}
\usepackage{pgfplots}
\usepackage{sansmath}
\usetikzlibrary{shapes.geometric}
\usepackage{wasysym}

\usetikzlibrary{calc}


%\theoremstyle{definition}
\newtheorem{aufgabe}{Aufgabe}
\newcommand{\Var}{\operatorname{Var}}
\newcommand{\D}{\operatorname{d}}
\newcommand{\mum}{\operatorname{\mu m}}
\newcommand{\E}{\operatorname{E}}
\newcommand{\var}{\operatorname{var}}
\newcommand{\Id}{\operatorname{Id}}
\newcommand{\rg}{\operatorname{rg}}



\renewcommand*\familydefault{\sfdefault}
%\renewcommand{\arraystretch}{1.1}


\KOMAoptions{parskip=half,DIV=15,fontsize=11pt}
\unitlength1cm
%\titlehead{
%
%\begin{center}\begin{tabular}{p{10cm}p{6.8cm}}
%\textbf{FH Aachen} & \textbf{FB Maschinenbau und Mechatronik}\\[0.5cm]
%83108: Betriebswirtschaft&  Prof. Dr. Raphael Pfaff\\
%und Technik der Eisenbahnen& Wintersemester 2015/15\\
%\end{tabular}
%
%\end{center}
%\begin{picture}(0,0)(0,0)\put(17,-23){\includegraphics[height=5cm]{fh_logo}}\end{picture}
%}
\newif\ifuelsg %als slides
%\uelsgtrue
\uelsgfalse
\newif\ifnotuelsg
\ifuelsg\notuelsgfalse\else\notuelsgtrue\fi

\title{\"Ubung ``Einf\"uhrung in den Zustandsraum''}
\date{}
\makeatletter
\let\Title\@title
\let\Author\@author
\makeatother
\pagestyle{fancy}
\fancyhead[L]{Prof. Dr. Raphael Pfaff}
\fancyhead[R]{85745: Energieeffiziente Antriebsregelung}
\fancyfoot[L]{Datei: \jobname}
\fancyfoot[R]{Datum: \today
\begin{picture}(0,0)(0,0)\put(.5,0){\includegraphics[height=5cm]{fh_logo}}\end{picture}}


\begin{document}
\vspace{-2cm}
\maketitle
\thispagestyle{fancy}
\vspace{-2cm}
% \hyphenation{Abwei-chungen}

\section*{Aufstellen Zustandsgleichungen}
\begin{aufgabe}[Zustandsgleichung eines RLC-Systems] 
\label{Task:RLCSystem}
Stellen Sie eine Zustandsraumbeschreibung des Systems wie in Abbildung \ref{Fig:RLCSystem} f\"ur den Zustandsvektor $\begin{pmatrix} 
		u_{C}(t) \\ \frac{1}{C} i(t)
		\end{pmatrix}$ auf.
\end{aufgabe}
\begin{figure}[htbp]
\begin{center}
\begin{circuitikz}[scale = 1]
		\draw 
	(-2,0) to  [V=$u(t)$] (-2,4)
		to [R=$R$,  i>^=$i(t)$] (2,4)
		to [L=$L$] (6,4)
		to [C=$C$, v_>=$u_{C}(t)$] (6,0)
  		 to [short] (-2,0);
		 \end{circuitikz}
\caption{RLC-System zu Aufgabe \ref{Task:RLCSystem}}
\label{Fig:RLCSystem}
\end{center}
\end{figure}

\vspace{0.5cm}
\begin{aufgabe}[Transformation \"Ubertragungsfunktion in Zustandsraumdarstellung]
Es sei \[G(s) = \frac{Y(s)}{U(s)} = \frac{s+3}{(s+1)(s+10)}\] die \"Ubertragungsfunktion eines Systems.
\end{aufgabe}
Stellen Sie $G(s)$ dar als:
\begin{enumerate}
		\item Differentialgleichung
		\item Zustandsraumdarstellung f\"ur einen Zustandsvektor $\begin{pmatrix} x \\ \dot{x} \end{pmatrix}$ in
		\begin{enumerate}
		\item Regelungsnormalform
		\item Beobachtungsnormalform
		\end{enumerate}
		\end{enumerate}
\vspace{0.5cm}
\begin{aufgabe}[Steuerbarkeit]
Gegeben sei das System
\begin{equation}
\label{Eq:SysCont1}
\dot{x}(t) = 
\begin{pmatrix}
 -1 & -1 \\ 1 & -3
\end{pmatrix} x(t) + 
\begin{pmatrix}
 1 \\ 1
\end{pmatrix} u(t), \quad x(0) = x_{0}
\end{equation}
\begin{equation}
\label{Eq:SysCont2}
y(t) = \begin{pmatrix}
 0 & 1
\end{pmatrix} x(t)
\end{equation}
\"Uberpr\"ufen Sie die Steuerbarkeit und Ausgangssteuerbarkeit des Systems (\ref{Eq:SysCont1}), (\ref{Eq:SysCont2}).

\end{aufgabe}

\vspace{0.5cm}
\begin{aufgabe}[Beobachtbarkeit]
Gegeben sei das System
\begin{equation}
\label{Eq:SysCont12}
\dot{x}(t) = 
\begin{pmatrix}
 0 & 1 \\ -1 & -3
\end{pmatrix} x(t) + 
\begin{pmatrix}
 1 \\ 2
\end{pmatrix} u(t), \quad x(0) = x_{0}
\end{equation}
\begin{equation}
\label{Eq:SysCont22}
y(t) = 
\begin{pmatrix}
1 & 1 
\end{pmatrix}
x(t)
\end{equation}
\"Uberpr\"ufen Sie die Steuerbarkeit, Ausgangssteuerbarkeit und Beobachtbarkeit des Systems (\ref{Eq:SysCont12}), (\ref{Eq:SysCont22}).
\end{aufgabe}

\end{document}